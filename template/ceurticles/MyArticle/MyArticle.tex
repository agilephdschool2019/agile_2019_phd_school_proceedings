\documentclass[a4paper]{article}
\usepackage{graphicx}
\usepackage{twocolpceurws}

% Added to support hyperlinks from rmarkdown
\usepackage[hidelinks]{hyperref}

\title{CEUR paper2p}

\author{

Mary Y. Writter

 \\ Publishing Dept.

 \\ Paper City, PUB 11011

 \\ \href{mailto:mqw@pubdept.eu}{\nolinkurl{mqw@pubdept.eu}}

\and


John X. Ceur

 \\ Science Dept.

 \\ Online City, CEUR 99099

 \\ \href{mailto:jqq@ceur-ws.org}{\nolinkurl{jqq@ceur-ws.org}}

\and


Yvonne Onderzoeker

 \\ Research Dept.

 \\ Science City, Sci 88088

 \\ \href{mailto:yvo@science.rdept.net}{\nolinkurl{yvo@science.rdept.net}}

\and

}

\institution{The Institute of CEUR Stuff}

\begin{document}
\maketitle

\begin{abstract}
The Abstract paragraph should be indented 1/4 inch (2 pica) on both the
left and right-hand margins. Abstract must be centered, bold, and in
point size 12. Two line spaces precede the Abstract. The Abstract must
be limited to one paragraph.
\end{abstract}

\hypertarget{general-formatting-instructions}{%
\section{General Formatting
Instructions}\label{general-formatting-instructions}}

The format is in one column. The left margin is .9 inches (5.5 picas).
Use 10 point type, with a vertical spacing of 11 points. Times Roman is
the preferred typeface throughout.

Paper title is 16 point, Caps/lowercase, bold, centered. Subsequent
pages should start at 1 inch (6 picas) from the top of the page.

Authors names are centered, initial caps; co-authors names, if used, are
flush left and flush right.

Paragraphs are indented by 1 pica with no space between paragraphs.

\hypertarget{first-level-heading}{%
\section{First Level Heading}\label{first-level-heading}}

First level headings are all flush left, initial caps, bold and in point
size 12. One line space before the first level heading and 1/2 line
space after the first level heading.

\hypertarget{second-level-heading}{%
\subsection{Second Level Heading}\label{second-level-heading}}

Second level headings must be flush left, initial caps, bold and in
point size 10. One line space before the second level heading and 1/2
line space after the second level heading.

\hypertarget{third-level-heading}{%
\subsubsection{Third Level Heading}\label{third-level-heading}}

Third level headings must be flush left, initial caps and bold. One line
space before the third level heading and 1/2 line space after the third
level heading.

\hypertarget{fourth-level-heading}{%
\paragraph{Fourth Level Heading}\label{fourth-level-heading}}

Fourth level headings must be flush left, initial caps and roman type.
One line space before the fourth level heading and 1/2 line space after
the fourth level heading.

\hypertarget{citations-in-text}{%
\subsection{Citations In Text}\label{citations-in-text}}

Citations within the text should indicate the author's last name and
year (Knuth 1973). Reference style Comer (1979) should follow the style
that you are used to using, as long as the citation style is consistent.

\hypertarget{footnotes}{%
\subsubsection{Footnotes}\label{footnotes}}

Indicate footnotes with a number\footnote{This is a sample footnote} in
the text. Place the footnotes at the bottom of the page they appear on.
Precede the footnote with a vertical rule of 2 inches (12 picas).

\hypertarget{figures}{%
\subsubsection{Figures}\label{figures}}

All artwork must be centered, neat, clean and legible. Do not use pencil
or hand-drawn artwork. Figure number and caption always appear after the
the figure. Place one line space before the figure, one line space
before the figure caption and one line space after the figure caption.
The figure caption is initial caps and each figure is numbered
consecutively.

Make sure that the figure caption does not get separated from the
figure. Leave extra white space at the bottom of the page to avoid
splitting the figure and figure caption.

Figure \ref{fig1} shows how to include a figure as PDF. The source of
the figure is in file \texttt{fig.pdf}.

\begin{figure}[ht]
\begin{center}
\includegraphics[height=4cm]{fig1}
\caption{Sample EPS figure }
\label{fig1}
\end{center}
\end{figure}

Below is another figure using LaTeX commands.

\begin{figure}[ht]
\begin{center}
\setlength{\unitlength}{1pt}
\footnotesize
\begin{picture}(160,80)
        \put(0,0){\framebox(160,80)[]{}}
        \put(10,35){\framebox(80,40){}}
        \put(100,20){\framebox(40,20){}}
        \put(70,10){\framebox(20,10){}}
        \put(20,5){\framebox(10,5){}}
\end{picture}
\caption{Sample Figure Caption}
\end{center}
\end{figure}

\hypertarget{tables}{%
\subsection{Tables}\label{tables}}

All tables must be centered, neat, clean and legible. Do not use pencil
or hand-drawn tables. Table number and title always appear before the
table.

One line space before the table title, one line space after the table
title and one line space after the table. The table title must be
initial caps and each table numbered consecutively.

\begin{table}[ht]
\begin{center}
\caption{Sample Table}

\bigskip

\begin{tabular}{|l|l|r|}
\hline
A & B & 1\\ \hline
C & D & 2\\
E & F & 3\\ \hline
\end{tabular}
\end{center}
\end{table}

\hypertarget{handling-references}{%
\subsection{Handling References}\label{handling-references}}

Use a first level heading for the references. References follow the
acknowledgements.

\hypertarget{acknowledgements}{%
\subsection{Acknowledgements}\label{acknowledgements}}

Use a third level heading for the acknowledgements. All acknowledgements
go at the end of the paper.

\hypertarget{references}{%
\section*{References}\label{references}}
\addcontentsline{toc}{section}{References}

\hypertarget{refs}{}
\leavevmode\hypertarget{ref-Comer-btree}{}%
Comer, D. 1979. ``The Ubiquitous B-Tree.'' \emph{Computing Surveys} 11
(2): 121--37.

\leavevmode\hypertarget{ref-Knuth-vol3}{}%
Knuth, D. E. 1973. \emph{The Art of Computer Programming -- Volume 3 /
Sorting and Searching}. Addison-Wesley.


\bibliography{samplebib.bib}

\end{document}
